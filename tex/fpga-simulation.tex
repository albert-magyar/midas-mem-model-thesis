To avoid confusion when speaking of computers simulating computers, the
literature commonly makes a distinction between the \emph{target}, the computer
being simulated, and the \emph{host} or \emph{host-platform}, the computer
executing the simulation. The host-platform is often not a single machine but
a collection of interconnected machines, which may include CPUs,
GPUs, and FPGAs.

\section{A Tour of Full-System Simulation}

Simulation performs three different functions.

\begin{enumerate}

    \item \textbf{\TODO{Better term - Prototyping}:} ``What thing should we
        build?" Prototyping serves as a means to rapidly evaluate different
        design points with an imperfect model of a proposed design.

    \item \textbf{Verification:} ``Did we build the thing right?" Verification
        serves to check (or perhaps, prove) that a particular implementation
        correctly executes.

    \item \textbf{Validation:} ``Did we build the right thing?" Validation
        serves to show that the implementation fulfills the objectives set out
        for the system.

\end{enumerate}

Both prototyping and verification can be applied at all levels of the design
hierarchy.  For example, given a specification of the system into which an
accelerator is integrated, one could prototype different design points and
verify an implementation of that accelerator. Validation, however, seeks to
answer a system-level question, that spans the entire computing stack.  The
surest way to validate a system is not in simulation, but with a physical
prototype or the final product itself, when the application can be run at
speed. This pushes validation late into the design cycle. To perform,
\emph{pre-silicon} validation a fast and accurate full-system simulator is
required.

%All simulation, full-system or otherwise, makes a tradeoff of between at least
%three competing objectives.  \emph{fidelity}~(how accurate is the simulation),
%\emph{speed}~(how quickly does simulator execute) and \emph{cost}~(\$ per
%simulation hour). \TODO{Table X lists some common technologies are illustrated
%in Table X}

Generally, there are only two points during the development of an SoC where
full-system simulation executes within even three orders of magnitude of a
silicon implementation. Very early, when architecture-level simulators are used
to perform initial system prototyping (hundreds of MIPS). And very late: when
the SoC is completely implemented and full-system emulation tools like a
Synopsys Palladium~\cite{palladium} (ones of MHz) can be used -- if they are
available.

When a hardware-emulation platform is unavailable or too expensive, and when
faster full-system simulation is desired earlier, it is common in industry to
use FPGA prototypes. An FPGA prototype directly implements the SoC on
one more more FPGAs, often with a custom board design that may include
peripherals identical to those that would be deployed in the final system. FPGA
prototypes are fast enough to support software development(ones to hundreds of
MHz) and thus a greater degree pre-silicon validation. Moreover, for large
project, they are inexpensive enough to duplicated and shared between hardware
and software engineering teams.  The difficulty with using FPGA prototypes as
full-system simulators is that they require complete RTL implementation of the
design (and, sometimes, a custom PCB). While FPGA prototypes can accelerate
software development by months, they become useful too late in the design cycle
to do hardware-software co-design, as the target hardware cannot easily be
modified, limiting the scope of potential changes.

Since a complete RTL implementation precludes the use FPGA prototypes for
design space exploration and early prototyping, they are generally unused in
academia. Instead, full-system simulation is done entirely in software with
simulation frameworks such as Gem5~\cite{gem5}, and MARSSx86~\cite{marssx86}.
These simulators can run target workloads at up to hundreds of KIPS, but are
often much slower in practice when employing detailed or custom models. This
makes it practically impossible to run complete workloads, such as
multi-threaded Java applications or SPECint2006~\cite{spec} with its reference
inputs. A common remedy is to employ statistical sampling
techniques~\cite{smarts} to fast-forward to the region of interest, before
executing O(100M) instructions at the desired fidelity.

While this approach has well-acknowledged shortcomings~\cite{gem5error},
judicious use of cycle-level simulators can be an appropriate vehicle for
proposing new microarchitectural ideas. But for radical proposals that involve
aggressive microarchitectural changes or traverse multiple layers of the
computing stack, this approach is inadequate (particularly for workloads that
are long-running, irregular and require a large number of cores, such as
managed-language workloads~\cite{MicroSimPanel}).

The ideal full-system simulator -- sufficient for both academic and industrial
uses -- would be inexpensive, fast and as accurate as desired throughout the
whole design process; it would always possible to run the software stack (or
what exists of it) on a model of the target hardware at speeds fast enough for
software development. Initially, this simulator could be used for system-level
prototyping and design space exploration, but as desired, more detailed models
or RTL implementations of target components would be integrated. Ultimately,
once all of the target RTL has been integrated, the simulator subsume the role of an
FPGA prototype. Academics need not completely implement an SoC; they may simply
stop adding fidelity to the simulation once they are content with the quality
of its results.

Presently, FPGAs are the only commercial off-the-shelf (COTS) technology
capable of supporting fast, scalable, cycle-accurate simulation. Thus, we
believe any attempt to build this ideal simulator must necessarily use FPGAs.
This will require a more flexible perspective of how FPGAs can be deployed as
simulation \emph{accelerators} and not merely RTL emulation devices as they are
used in FPGA prototypes.

\section{Defining FPGA-Accelerated Simulation}

The distinction between FPGA-accelerated simulation and FPGA prototyping can be
nebulous: we argue that FPGA prototyping represents a subset of the larger
space of FPGA-accelerated simulation. Key to understanding this is first, think of
simulation as any other application executing on a host, and second, forget
that a host may be or include one or more FPGAs.

A simulation is an application that takes an input and produces an output.  The
output may be the console or file I/O of the target. In other cases, it
may be a log of all memory requests made to DRAM or instructions retired.
Alternatively, it may be a dump of the microarchitectural state of the target as it changes over the
lifetime of the simulation.  As far as the user is concerned, the simulation
may be optimized in any way whatsoever so long as this output is the same.

Like any other application, simulations have hot spots that account for the bulk
of their runtime. To improve runtime, the simulation may be parallelized over
the host, either over multiple homogeneous resources, or by offloading specific
kernels to accelerators, which may execute concurrently or with the rest
of the application.

%In simulations that account for time at the cycle-level, much of this runtime
%is dedicated to modeling the cycle-by-cycle interactions of parts of the
%system. These models may be written in C++ or SystemC, or implemented in an HDL
%like verilog or VHDL. Parts of the simulation may be divided into functional
%models, like an architecture simulator, and a timing-model that does some
%accounting of time based on a model of the microarchitecture. What's important
%to note, is that any cycle-level model of hardware attempts to capture the
%behavior of a fine-grained highly concurrent digital-circuit. Taken to the
%limit, a cycle-level model is an RTL model.

In FPGA-accelerated simulation, in general, we attempt to offload parts of the
simulation that do cycle-level or cycle-accurate modeling of some part of the
target. One way to achieve this is to dissolve the simulation spatially
(perhaps along module boundaries): parts of the target, like an NoC or Core
pipeline, could be offloaded to an FPGA, while models for I/O may be hosted on
a CPU.  Alternatively, one could host a functional model of a module on the CPU
and accelerate the timing-model on the FPGA (or vice versa).  Again, the only
constraint on the implementation of the simulation, and thus, the
implementation of FPGA-accelerated components of the simulation, is that as the
output of the simulation is the same. Once this constraint is met, a faster
implementation is always better.

\section{Adoption Challenges}

Despite their promises, FPGA-accelerated simulation has only been successfully
employed by the researchers that developed them. The failure to adopt
FPGA-accelerated simulation methodologies more widely comes as a result of
several key factors:

\begin{enumerate}

    \item \textbf{Availability.} Much of the early FPGA simulator research
        relied on boutique FPGA-host platforms like the BEE~\cite{bee2}, or
        used custom board designs. The cost of these platforms disincentivizes
        their adoption by researchers who already have the means to run
        software simulations at low cost.

    \item \textbf{FPGA Capacity.} Common ASIC structures, such as CAMs and
        multi-ported RAMs, are known to map poorly to FPGA
        fabrics~\cite{fpgagap, fpgagap2}, making it difficult to host large ASIC
        designs on an FPGA.

    \item \textbf{Configurability \& Extensibility.} Extending FPGA-accelerated
        simulations requires writing RTL. Models written in RTL tend to be less
        configurable and are harder to extend than an equivalent software
        model. Finally, RTL models still need to be validated, further
        exacerbating the challenging of building them.

    \item \textbf{FPGA compile time.} Compiling an FPGA simulator takes many
        orders of magnitude longer than compiling a software simulator (ones of hours).
        %To some extent this is inescapable. However, where abstract models are
        %employed, they can be made run-time configurable, with programmable
        %registers sitting on a simulation memory map. Where models are
        %generated from RTL, they can can be incrementally recompiled, or perhaps
        %partially reconfigured.

    \item \textbf{Debuggability.} Debugging a broken FPGA-accelerated
        simulation is difficult due to the limited visibility the designer has
        over the state of the simulation. This is often more challenging than
        debugging a an FPGA prototype of the target, as FPGA-specific
        optimizations make it more difficult to reason about the state of the
        target.

\end{enumerate}

\section{Why Revisit FPGA-Accelerated Simulation?}

Even as Moore's law wanes, FPGA capacity continues to scale. The largest FPGAs
have over 50 MB of BRAM and millions of logic cells\footnote{Comically, scaling
RAMPGold~\cite{rampgold}, to use the largest Xilinx UltraScale
FPGA~\cite{ultrascale} by BRAM capacity would permit modeling in excess of 5000
cores.}. As they have scaled, FPGAs have continued to become more
heterogeneous, adding features that make them more amenable to hosting
full-system simulators.  Both Intel and Xilinx now sell FPGAs with embedded ARM
cores, making it easier to co-simulate tightly coupled hardware and software
models of a system. Modern FPGAs include hardened DRAM controllers that are
comparable to those of ASICs. This trend towards greater integration looks to
continue. For example, upcoming Intel Stratix 10 MX parts include in-package DRAM (HBM2)
that can support up to 1 TBps of aggregate memory bandwidth~\cite{stratix10mx}.
Both DRAM capacity and bandwidth are crucial for simulating components
of the target that may not fit in BRAM.

Lower cost and increased on-chip integration have also made FPGAs more
accessible. Not only are COTS development boards cheaper and more featureful,
FPGAs are now available as a service, both through academic clusters, like
TACC's Microsoft Catapult~\cite{catapultannounce} deployment, and through Amazon
Web Services' EC2 F1 instances~\cite{amazonf1}. Where in the past academics
would have to purchase their own FPGAs -- even to reproduce published
experiments -- it may soon be possible for them to instead spin up an identical
simulation on a shared computing resource. Companies may no longer have to
maintain their own FPGA prototyping clusters; they could instead batch out
simulations to the cloud.

\section{Improving Usability Through Automation}

While the trends described in the previous section solve the
\emph{availability} and \emph{FPGA capacity} challenges, the usability
challenges remain. Previous work~\cite{fabscalarfpga, strober} has shown that
much of an FPGA accelerated simulation can be automatically generated from
source RTL. This RTL can be written in an HDL like Verilog or emitted by
high-level synthesis tools or generators written in languages like
Chisel~\cite{chisel}. While this still requires an RTL implementation, the same
RTL to measure physical design metrics, and no validation of the generated
model is required.

This is not a panacea as it is sometimes difficult or undesirable to
automatically generate models from source RTL: perhaps the RTL is not yet
available, or a more abstract, reconfigurable model is desired. In this report
we consider off-chip DRAM memory systems as a motivating example: they have too
much state to be hosted in fabric and yet they must must be tightly coupled to
the processor model making it difficult to co-simulate DRAM in
software.\footnote{High-latency peripherals, like disks, can often be modeled
in software without any performance cost~\cite{disksim}.} These components
typically require an abstract model that virtualizes the target-memory system
over FPGA-host-memory system.

This reintroduces the aforementioned problem that anything but a simplistic RTL
model is difficult to design, modify and reuse. We propose to address this
through \emph{generators} that synthesize abstract memory system models that
can be easily modified and used across a wide range of targets.  This generator
must provide a variety of timing-models so as to enable the designer to trade
fidelity for FPGA area when needed. Generated models must be reconfigurable, to
permit sweeping memory system parameters without needing to recompile the FPGA
bitstream. These models must provide useful instrumentation to both aid in
debugging and to provide insight about memory system behavior without
perturbing execution. Finally, when the generator does not provide a timing
model fit for the user, the generator should be easy to extend.
