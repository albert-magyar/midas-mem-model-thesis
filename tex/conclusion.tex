While FPGA-accelerated simulation is not yet a standard technique for
full-system simulation, it appears to be the only path for obtaining the fast
and accurate full-system models required to investigate many future computing
applications and platforms. The advent of cloud-hosted FPGA-enhanced server
instances significantly reduces the barrier to using FPGA-accelerated
simulation, but productive development of full-system FPGA simulations is still
a challenge.

MIDAS is making inroads to solving these challenges by automating the
construction of FPGA-accelerated simulations. Automatically deriving
cycle-exact, bit-exact FPGA models from synthesizable target RTL modules
reduces model-building and validation effort and enables researchers and system
designers to also observe cycle-time, area, and power impacts using commercial
ECAD tools.

However, when RTL implementation has not yet been built, or when that RTL is
not easily tranformed into an FPGA model, parameterized generators to produce
configurable models for these components are required. This report demonstrates
this approach for modelling off-chip DRAM memory systems. Combining these two forms
of model generation provides a highly productive environment for performing
complex computer architecture research, as shown by the detailed results
obtained for garbage collection in a managed-language runtime.
