We provide two DDR3 timing-model classes based around FCFS and FR-FCFS MAS.
DDR3 timing models have a runtime-configurable address assignment, speed grade,
page policy, and rank, bank, and row count. The timing models generate legal
DDR3 command traces that have been validated against Verilog golden models.

\section{Timing Model Design}
Both timing-model classes consist of four components: transaction
scheduler, DRAM state trackers, MAS, and backend.

\subsection{Transaction Scheduler}
The transaction scheduler consists of a single, unified, configurable-depth
queue that can accept an AXI4 read and write transaction simultaneously. Reads
are given priority when only one slot is available.  Transactions are passed to
the MAS as they can be accepted at a rate of one transaction per cycle.

\subsection{DRAM State Trackers}
We decoupled the MAS model design from structures that track the state
of DRAM devices. The DRAM state tracker is arranged hierarchically into rank and
bank state trackers. State trackers present to the MAS a bit vector indicating
which commands they may legally accept. Trackers have a counter for each command
type, which indicate the next earliest cycle that the tracker can legally accept a command of
that type. When the MAS issues a command, it informs the associated state tracker,
which updates its counters accordingly.

\subsection{Memory Access Scheduler}
Each MAS maintains a data structure of memory references it is scheduling across.
When a new transaction is received, a record in this structure is populated with the decoded rank, bank, and row addresses, along with
MAS-specific metadata to ease scheduling decisions.
On every target cycle, the MAS
selects a legal command based on the available references and the bit vectors
presented by the state trackers. The MAS releases reads to the backend on the
cycle the first beat would return from DRAM. Writes are released the cycle after a
CASW command is issued.  Both MAS models have a simple, runtime-configurable
page policy. The open-page policy keeps pages open until the next refresh or another row
is to be opened. The closed-page policy always issues auto-precharged CAS
commands. To maintain a global age-order of memory references, the FR-FCFS MAS
uses a single collapsing buffer with a runtime-programmable depth.

\subsection{Refresh Policy}
Both MAS use an interrupting, all-ranks refresh scheme that makes no attempt to
pull-in or delay refresh commands.  When a refresh is requested, the MAS
precharges all banks in all ranks, and issues REF commands as soon as possible.

\subsection{Backend}
The backend receives read and write references from the MAS and drives the AXI4
response channels back to the target. If no LLC is being used, read response data is fetched from the host-response staging unit.
The backend can apply an additional runtime-configurable latency to simulate
additional latency on read responses and write acknowledgments.

We give the FPGA resource utilization for a handful of representative DDR3
instances in Table~\ref{tbl:dram-model-resources}.

\begin{table}[htb]
\centering
    \begin{tabular}{c c c c c}
	\hline
        \textbf{Example Instance} & LUTs & FFs & BRAM & $f_{MAX}$ \\
	\hline
        FCFS Single Rank~(SR)    & 2272 & 1647 & 1 & 272 \\
        FCFS Quad Rank~(QR)    & 4433 & 3263 & 1 & 219 \\
        FR-FCFS, SR, 8 Deep & 3172 & 2188 & 5 & 239 \\
        FR-FCFS, QR, 16 Deep & 8206 & 4805 & 1 & 158 \\
	\hline
	\end{tabular}
    \caption{Resource counts and best-case $f_{MAX}$ for four different DDR3
    models. The last instance can schedule over 16 references and thus has a larger functional model.}
\label{tbl:dram-model-resources}
\vspace{-0.35in}
\end{table}

\section{Validation}
Since our DRAM-timing models generate DRAM-command traces, we validated
traces collected in RTL simulation against a Micron DDR3 golden model. The golden
model simulates a single-device slice of the memory organization and detects
DRAM timing violations in the command trace.  We made small modifications to
the provided test bench to support validation of multi-rank organizations and
to accept our trace format. This is the same validation approach used by all
popular software cycle-accurate simulators.
To perform performance validation, we generated memory-reference traces for
which we could estimate bandwidth and row buffer hit rates a priori. The traces
were tailored to expose different memory-access scheduling behavior in the MAS
models.

\section{Selecting Legal Runtime Configurations}
Our most sophisticated DDR3 timing-model instances expose over thirty
runtime-programmable settings that specify the low-level DRAM timings,
address-assignment scheme, LLC-model configuration, and MAS-structure sizes.
Each of these settings has a legal range of values that depends on hardware in
the generated instance. To make it easier to program instances, we provide a
runtime-configuration utility that assigns these settings
from higher-level free parameters.
Users specify the DRAM data-bus width,
device DQ width, number of ranks, and total DRAM capacity. Our utility
then searches for a DDR3 device in a database and chooses one with an appropriate
density and speed grade for that memory system.

\section{Comparison to DRAMSim2} \PNAME models nearly all of the same
aspects of DRAM as DRAMSim2, but it is lacking support for burst
chopping and for power-down and self-refresh modes.
We plan to add these features in the future. \PNAME does not natively model
multi-channel memory systems, as this can be easily accomplished by using
multiple instances. Similarly, \PNAME does not natively simulate a clock-domain
crossing between the controller and the rest of the processor, instead, this is modeled in the MIDAS-generated channels.  \PNAME makes up for these
limitations with simulation speed: as part of a full-system
simulator \PNAME executes  100x - 1000x faster than popular cycle-accurate
simulators running standalone (measured in~\cite{ramulator})! We expand on
\PNAME's performance in Section~\ref{sec:evaluation}.
