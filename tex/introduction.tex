For the past half century, transistor scaling has delivered incredible
improvements in computing performance and energy efficiency. The rapid advance
of process technology has proven to be a strong disincentives to building custom ASICs and SoCs, 
as the competitive advantage of a custom silicon design would usually be lost
to a general-purpose machine built in a newer process technology. This coupled
with famously large and growing NRE, has limited ASIC deployment to high volume
domains where they offer a considerable advantage over a general-purpose part.
For applications that would see enormous improvements with custom silicon but
lack the volume to overcome the NRE, FPGAs have proven to be an effective
stopgap, despite being inferior to an ASIC in logic density, delay, and energy
efficiency.

With Moore’s law ending, the risk balance for system architects is beginning to
change: while NRE remains high, custom silicon may now present a lasting
competitive advantage.  Application, compiler, and OS developers will be more
inclined to support specialized devices if they can deliver a performance
improvement that cannot foreseeably be met by a general purpose machine.
Perhaps the greatest performance and energy efficiency improvements are to be
had when hardware and the software can be \emph{co-designed} for one another.
In the absence of frequency scaling, custom silicon and hardware-software
co-design are perhaps the only ways to maintain the advances in computing
performance and efficiency advances the market has come to expect.

Unfortunately, the NRE of custom silicon remains a barrier to custom silicon.
NRE is driven by many factors, including mask, EDA, and IP costs, as well
engineering time for design and verification. Increasingly, however, the
largest contributor to NRE is that of software development.  Ideally, software
development would proceed in parallel to hardware design, however, lacking
fast, cycle-accurate \emph{full-system simulators} (simulators capable of
executing the entire software stack), software engineers are forced to develop
against a inaccurate model of the SoC or wait until first silicon is back.
Moreover, digital logic bugs that would be caught by a long running full-system
simulation are not detected, forcing costly silicon respins.  Finally, this
serialization also precludes effective hardware-software co-design, as it
difficult to reason about full-system design tradeoffs if the software and
hardware are being evaluated in different simulation environments. This lack of
fast and accurate full-system simulation is often referred to as the
\emph{simulation gap}.

The simulation gap has also plagued academic research. Software simulations
have long been the mainstay of the computer architecture research community,
but these have struggled to provide meaningful results especially since the
spread of multicore systems in the mid-2000s.  While a number of abstracted and
sampled simulation techniques have been developed to help reduce the runtime of
software simulations, these generally cannot cope with dynamic workloads, such
as JIT compilers, where code paths observed vary depending on performance, and
where long runtimes are needed to observe meaningful application behavior.  As
a result, few architectural studies use managed languages, and are often based
solely on small inputs to statically compiled SPEC codes.  Furthermore,
software simulations are notoriously inaccurate and have no direct path to
obtain accurate physical metrics such as cycle time, area, or power.

The recent trend towards heterogeneous SoCs with a plethora of custom hardware
accelerators for both server and mobile applications has only exacerbated the
simulation gap. There is a belief that new programming models, runtimes, and
operating systems will be necessary to ease the challenge of programming
machines that may have dozens of custom accelerators. This work ranges from new
OS designs such as data-plane operating systems~\cite{arrakis} and
multikernels~\cite{barrelfish} to a renewed interest in managed-language
runtime systems such as Java Virtual Machines~\cite{broom,taurus}.

While research employing hardware-software co-design of such systems has the
potential to yield substantial improvements, it is challenging with current
simulation methodologies.  Many of these systems have the property that
workloads are long-running and irregular (e.g., due to garbage collection or
JIT compilation), but the interactions between the application and system layer
are often very fine-grained (such as a few cycles spent in an interrupt handler
or a call into the memory allocator).  Fast, cycle-accurate, full-system
simulation is desired to properly account for the interactions between the
layers of the software stack and the hardware underneath.

To address the simulation gap, many have turned to FPGAs. In industry,
companies have long relied on \emph{FPGA-prototypes} that implement the SoC's
RTL directly. These prototypes are both relatively cheap to build (considering
total cost of the project) and fast enough to support software development
(executing at ones to tens of MHz). For academics, FPGA prototypes are too
expensive and inflexible as they require a complete RTL implementation of the
SoC. Instead, academia has explored using FPGAs as \emph{accelerators} for
architecture simulations~\cite{fast, fame, hasim, protoflex,ramp} that are
better suited for prototyping research ideas and performing design space
exploration. \emph{FPGA-accelerated simulation} has seen little adoption in the
academia or industry for a number of reasons: 1) FPGA-accelerated simulators
are difficult to write or modify, 2) FPGA mappings require a lengthy
compilation time, 3) FPGAs are difficult to debug, 4) FPGAs have historically
been resource constrained, either limiting the scale of the system under
simulation or requiring even more complexity to partition designs across FPGAs,
and 5) FPGA hardware and software tools are expensive to purchase and maintain.

Recent technological advances have eased some of these challenges of employing
FPGA-accelerated simulators. FPGAs are larger and faster than ever before and
are becoming available as resources in research clusters\cite{catapultannounce}
and datacenters\cite{amazonf1}.  Given the lack of a competing technology that
is both as cheap and as effective at accelerating cycle-accurate simulations,
it appears that FPGAs remain the only vehicle capable of bridging the
simulation gap, and thus demand a careful reconsideration.

\emph{MIDAS} (Modeling Infrastructure for Debugging and Simulation) is Berkeley
Architecture Research's answer to improving the usability of FPGA-accelerated
simulators. MIDAS permits co-hosting software models and FPGA-accelerated
models on an arbitrary \emph{host-platform} consisting of a mix of FPGAs and
CPUs.  During early design space exploration or system specification, a MIDAS
simulation may consist entirely of software models of the system.  However, as
RTL implementations become available, MIDAS can automatically transform them
into FPGA-accelerated models that can be linked into the simulator. Once the
SoC has a complete RTL implementation, a MIDAS generated simulator can subsume
the function of an FPGA-prototype. Throughout the whole process there is a
functioning model of the machine, that is fast enough for software development.

One challenge with this vision, is that there are many cases in which an
FPGA-accelerated model is required but there is not yet RTL available to be
transformed. Here custom-RTL models are required and are subject to the same
usability pitfalls of prior FPGA-accelerated simulation work. For MIDAS to
succeed, models like this must be flexible enough to be reusable across a wide
range of simulations in order to amortize the high cost of writing them.
Generally, I propose that using a hardware \emph{generator} to synthesize
timing model \emph{instances} addresses this challenge. To this end, the
generator described herein synthesizes models for an important component of
nearly all SoCs: DRAM off-chip memory systems. These model instances use the
available FPGA off-chip memory system as a backing store for a timing model
hosted in fabric. These instances are reconfigurable at runtime, allowing for
design space exploration without lengthy FPGA recompilation. The generator can
also add instrumentation to ease debugging and to perform non-invasive
measurements of the simulation. Finally, since the in-fabric timing model is
split from the \emph{functional} units that interact with the FPGA memory
system, extending the generator with new timing-model classes is made easy.

This report builds on the work of an earlier masters thesis by Asif
Khan\cite{khanmasters} addressing challenge of modeling off-chip memory
systems. This report differs from that work, in its use of a generator, its
demonstration of the model interoperating with models automatically transformed
from source RTL, and finally, in the degree to which off-chip memory systems
are modeled.

\section{Collaboration, Previous Publications, and Funding}

This report builds extensively on the work of current and previous students of
our group. The MIDAS project is a multi-student collaboration between myself,
Jack Koenig, Sagar Karandikar, Deborah Soung, and Donggyu Kim, whose Chisel3
port of Strober\cite{strober} project constitutes the bulk of the MIDAS code
base at the time of writing. The generator presented here and of the RTL used
in MIDAS is written in Chisel3\cite{chisel}, which remains under active
development with the guidance from SiFive and Google. MIDAS leans heavily on
FIRRTL\cite{firrtl}(Adam Izraelevitz et al.) to transform source RTL into
FPGA-accelerated models. Finally, MIDAS depends on the availability of
open-source SoC IP as a source of transformable RTL. Here we use the
Rocket-Chip\cite{rocketchip} SoC generator and Chris Celio's BOOM(\cite{boom})
OoO core generator which implement the free and open RISC-V ISA.

This report borrows heavily from our MICRO50 submission on the same subject.
Most of the results were collected by Donggyu Kim. Our infrastructure to
automatically build Linux images containing the desired workload and submit
jobs to our local FPGA cluster was initially written by Chris Celio. Martin Maas
was responsible for all things Java: porting the JikesRVM, and for
instrumenting Java applications to produce the plots in the case study.

\TODO{Funding?}
\TODO{What more to say about MICRO publication?}


