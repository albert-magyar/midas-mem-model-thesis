\chapter{Introduction}

For the past half century Moore's law, and Dennard's law before it, has
delivered incredible improvements in computing performance and energy
efficiency. The rapid advance of process technology has proven to be a strong
disincentive to building ASICs, as the competitive advantage of a custom
silicon design would usually be lost to a general-purpose machine built in a
newer process technology. This coupled with growing non-recurring
engineering~(NRE) costs, has limited ASIC deployment to high volume domains
where they offer a considerable advantage over a general-purpose part. For
applications that would see enormous improvements with custom silicon but lack
the volume to overcome the NRE costs, FPGAs have proven to be an effective
stopgap, despite being inferior to an ASIC in logic density (40x), frequency
(10x), and energy efficiency.

With Moore’s law ending, future advances in performance and energy-efficiency
must come from improvements above the process technology. For system designers
the risk balance is changing: while the NRE costs remain high, custom silicon
may now present a lasting competitive advantage. Application, compiler, and OS
developers will be more inclined to support more specialized computing devices
if those devices can deliver a performance improvement that cannot be met by a
general purpose machine. Perhaps the greatest opportunity for innovation, both for the
deeply embedded systems of the Internet-of-Things and the for the warehouse
scale computers of the Cloud, lies in hardware-software \emph{co-design}.

Unfortunately, the high NRE costs custom silicon remain. NRE costs are driven
by many factors, including mask, EDA tool, and IP costs, as well engineering
time for design and verification. However, the largest contributor to NRE is
often that of software development, especially in embedded systems. Ideally,
software development would proceed in parallel to hardware design, however,
lacking fast, cycle-accurate \emph{full-system simulators} (simulators capable
of executing the entire software stack), software developers are forced to
develop against a inaccurate model of the SoC or wait until first silicon is
back. Logic bugs that would be caught by a long running full-system simulation
are not detected, forcing costly silicon respins. This serialization
also precludes efficient hardware-software co-design, as it difficult to reason
about full-system design tradeoffs if the software and hardware are being
evaluated in different simulation environments. Instead, poor design decisions
based on poor system modeling lead to more silicon respins and failed products.

Research in academia too, has been hampered by the lack of fast,
cycle-accurate full-system simulation.  Software simulations have long been the
mainstay of the computer architecture research community, but these have
struggled to provide meaningful results especially since the spread of
multicore systems in the mid-2000s.  While a number of abstracted and sampled
simulation techniques have been developed to help reduce the runtime of
software simulations, these generally cannot cope with dynamic workloads, such
as JIT compilers, where code paths observed vary depending on performance, and
where long runtimes are needed to observe meaningful application behavior.  As
a result, few architectural studies use managed languages, and are often based
solely on small inputs to statically compiled SPEC codes.  Furthermore, software
simulations are notoriously inaccurate and have no direct path to obtain
accurate physical metrics such as cycle time, area, or power.

The recent trend towards heterogenous SoCs with a plethora of custom hardware
accelerators for both server and mobile applications has only exacerbated the
simulation gap. There is a belief that new programming models, runtimes, and
operating systems will be necessary to ease the challenge of programming
machines that may have dozens of custom accelerators. This work ranges from new
OS designs such as data-plane operating systems~\cite{arrakis} or
multikernels~\cite{barrelfish} to a renewed interest in managed-language
runtime systems such as Java Virtual Machines~\cite{broom,taurus}.

While research employing hardware-software co-design of such systems has the
potential to yield substantial improvements, it is challenging with current
simulation methodologies.  Many of these systems have the property that
workloads are long-running and irregular (e.g., due to garbage collection or
JIT compilation), but the interactions between the application and system layer
are often very fine-grained (such as a few cycles spent in an interrupt handler
or a call into the memory allocator).  Fast, cycle-accurate, full-system
simulation is desired to properly account for the interactions between the
layers of the software stack and the hardware underneath.  there is a belief
that new programming models, runtimes, and operating systems will be necessary
to ease the challenge of programming machines that may have dozens of custom
accelerators. This work ranges from new OS designs such as data-plane operating
systems~\cite{arrakis} or multikernels~\cite{barrelfish} to a renewed interest
in managed-language runtime systems such as Java Virtual
Machines~\cite{broom,taurus}.

To address this simulation gap, many in industry and some in academia have
turned to FPGAs. In industry, companies have long relied on
\emph{FPGA-prototypes} which consist of one or more FPGAs on a custom PCB that
implement the SoC's RTL directly. These prototypes and are both relatively
cheap to reproduce (considering total cost of the project) and fast enough to
support software development (executing at ones to tens of MHz). For academics,
FPGA prototypes are expensive and inflexible (as they require a complete RTL
implementation). Instead, earlier academic work explored using FPGAs as
accelerators for architecture simulations~\cite{fast, fame, hasim,
protoflex,ramp}.  But these approaches have seen little adoption in the
architecture community or industry for a number of reasons: 1) FPGA-accelerated
simulators are difficult to write or modify, 2) FPGA mappings require a lengthy
compilation time, 3) FPGAs are difficult to debug, 4) FPGAs have historically
been resource constrained, either limiting the scale of the system under
simulation or requiring even more complexity to partition designs across FPGAs,
5) the cost of purchasing and maintaining FPGA hardware and software tools.
Recent technological advances have eased some of these challenges of employing
FPGA-accelerated simulators. FPGAs are larger and faster than ever before and
are becoming available to researchers as resources in research
clusters\cite{catapultannounce} and datacenters\cite{amazonf1}.  However, the
usability challenges remain.

\emph{MIDAS} (Modeling Infrastructure for Debugging and Simulation) is Berkeley
Architecture Research's answer to the full-system simulation gap.  Building off
the findings of the RAMP\cite{ramp} project at Berkeley, MIDAS differs in
primarly in it's objective aims to make RAMPGold\cite{rampgold} and
DIABLO\cite{diablo}-like simulators easier to build, use and extend. MIDAS
permits co-hosting SW models and FPGA-accelerated models on an arbitrary
\emph{host-platform} consisting of a mix of FPGAs, CPUs. As implementations, or
more complete models of components of the system become available, they can
replace more abstract ones, easing the transistion from an abstract prototype
to a simulator capable of pre-silicon validation. Throughout the whole process
there is a functioning model of the target machine, that is fast enough for
software developers and researchers to code against. We believe this methodolgy
can empower both academic and industrial design efforts, and encourage greater
technology transfer from academia to industry.

This report presents an important component of the MIDAS environment: an
FPGA-hosted off-chip memory timing model, that is both reconfigurable and
agnostic of the particular FPGA architecture that hosts it. Just as MIDAS
succeeds RAMP, this report succeeds an earlier masters thesis by Asif
Khan\cite{khanmasters} on the same subject. This report differs from that work
in three dimensions: 1) The use of a generator to flexibly emit different
instances of a timing model. 2) Demonstration of the ability to integrate this
model with models automatically transformed from source RTL. 3) The degree to
which off-chip memory systems are modeled.

\section{Collaboration, Previous Publications, and Funding}

In what I still believe is generally good advice, my undergraduate advisor
Jonathan Rose once suggested to me that one should not couple the success of
one research project to that of another.\\

At Berkeley, we vehemently eschew this advice.\\

This report builds on the work of current and previous students of BAR.  The
MIDAS project is a multi-student collaboration between myself, Jack Koenig,
Sagar Karandikar, Deborah Soung, and Donggyu Kim, whose Chisel3 port of
Strober\cite{strober} project constitutes the bulk of the MIDAS code base at
the time of writing. All of the RTL used in MIDAS and this work is written in
Chisel3\cite{chisel}, which remains under active development with the guidance
from SiFive and Google. MIDAS leans heavily on FIRRTL\cite{firrtl}(Adam
Izraelevitz et al.) to transform source RTL into FPGA-accelerated models.
Finally, MIDAS depends on the availability of open-source SoC IP. Here we use
the Rocket-Chip\cite{rocketchip} SoC generator, and Chris Celio's
BOOM(\cite{boom}) OoO core generator.

This report borrows heavily from our MICRO50 submission, on the same subject.
Most of the results were collected by Donggyu Kim, borrowing some
infrastructure initially built by Chris Celio to automatically build linux
images and submit jobs to our local FPGA cluster. Martin Maas was responsible
for all things Java: porting the JikesRVM, and for instrumenting Java
applications to produce the plots in the case study.

\TODO{Funding?}

\TODO{What more to say about MICRO publication?}


